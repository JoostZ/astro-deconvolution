\section{Introduction}\label{index_introduction}
Stars are in principle point sources. However, if we take a picture of stars, we find that the star images have a finite extension. This is caused by the atmosphere and imperfections of the telescope and recording media. This spreading of the star image can be described by {\itshape convolution\/}: \[I(x, y) = \int _{-\infty} ^ {+\infty} \int _{-\infty} ^ {+\infty} PSF(x', y') S(x-x', y-y')dx' dy'\] where $S(x,y)$ is the real star image, $PSF(x,y)$ is the {\itshape Point Spread Function\/} and $I(x,y)$ is the resulting image.

Programs like IRIS allow you to do {\itshape deconvolution\/}, i.e., to reverse the process described above and to get an image that is a better representation of $S(x,y)$. However, this method is restricted to \%PSF that are position independent. Some of the distortions that can occur in image recording have a \%PSF that is position dependent. For instance, coma is a deformation that increases to the edge of the picture. Another example is a picture taken with a telescope mount where the polar axis is not perfectly aligned, causing stars to be recorded as circle segments.

This program is based on the article {\tt Deconvolution with a spatially-\/variant PSF}.\section{The theory}\label{index_theory}
A standard way of doing deconvolution is the {\itshape Lucy-\/Richardson Deconvolution Algorithm\/}. This algorithm starts with an original estimate of the star image $P(x,y)$. This image is then convoluted with the known PSF and the result is compared with the actual image $I(x,y)$. Based on the difference a new estimate of the star image is generated and these steps are repeated until the two images compare well.

By far the fastest way to do the convolutions is by means of the Fast Foerier Transform (FFT). However, this only works the PSF is constant over the area of the image.

As described in {\tt Deconvolution with a spatially-\/variant PSF} it is possible to generate a set of orthogonal base PSF, analogous to generating a set of orthogonal base vectors in an N-\/dimensional vector space, in such a way that each PSF can be expressed as a lineair combination of these base PSF. If we write the base PSF, that are independent of the position in the image as $P_i(x,y$, then we can write \[ P(u, v,x,y) = \sum _ {i = 1} ^ N a_i(u, v, x,y) P_i(x,y) \] where $u$ and $v$ are the coordinates in the image and $x$ and $y$ are trhe coordinates in the PSF.

With this expression we can write the convolution of the star image with the spatially variant PSF as \[ I (x,y) = \sum _{i = 1} ^ N \int _{-\infty} ^ {+\infty} \int _{-\infty} ^ {+\infty} S(u, v) a_i (u, v) p_i (x -u, y - v)du dv \]

Each of the terms in this formula is now a convolution with a PSF that is spatially invariant and and can therefore be calculated with FFT.

This program handles correction of (astro) photos by deconvolution with a spatially-\/variant PSF.\section{The Algorithm}\label{index_algorithm}
The first step is to determine the \%PSF as function of the position in the image. This program does not use theoretical formluas ot parameterized functions for the \%PSF. In stead it assumes that the pixels around a star in the recorded image {\itshape is\/} the PSF. This is done in class PSF. 